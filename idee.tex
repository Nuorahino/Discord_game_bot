\documentclass[fleqn,a4paper,11pt,leqno]{scrreprt} %Definiert den Dokumenttyp (hier braucht Ihr nichts ändern)
\usepackage{/home/georg/Uni-Heidelberg/ss20/Vorlagen/HA_preamble} %lädt eine Reihe von weiteren Paketen, Definitionen und Befehlen aus der Datei "HA_preamble.sty". In der Formulierung muss die Datei im selben Verzeichnis sein wie die Datei hier. Ihr könnt aber auch relative und absolute Pfade, z. B. "../../LaTeX/HA_preamble", nutzen, was auf Dauer praktischer ist.


%%%%%%%%%%%%%%%%%%%%%%%%%%%%%%%%%%%%%%
% Hier könnt Ihr weitere Packages laden, Befehle definieren etc.
%%%%%%%%%%%%%%%%%%%%%%%%%%%%%%%%%%%%%%

\begin{document}
\thispagestyle{firstpage}
\setcounter{sheetnr}{01} %hier müsst Ihr die Nummer des aktuellen Blattes hinschreiben

%%%%%%%%%%%%%%%%%%%%%%%%%%%%%%%%%%%%%%
% INPUT GOES HERE
%%%%%%%%%%%%%%%%%%%%%%%%%%%%%%%%%%%%%%
%Begin Input {sol0.tex
\begin{itemize}
  \item Implement the card deck, as arrays, or stack of card elements, or simpler strings
  \item shuffle the deck using shuffle from random
  \item draw card simply takes the bottom or top cards
  \item Discard pile works the same as draw pile, when reshuffeling, sum both decks and shuffle
  \item Players who join get a role "active player" on the server,
    when dying etc. this role gets removed
  \item Commands can be send in a server chat, or in private
  \item The bot respondes Error Messages in the same chat, and game info in a designated chat
  \item Info which not everybody should see is written in private
  \item Everybody can start a game, if the bot is in state idle, while the bot is in a game, no
    new game can be started
  \item only Admins, and game masters, can end games prematurely, or potentially by majority vote
  \item Bot should be able to return, what it is waiting on / current status
  \item At the end of the game, the bot should return to idle
  \item Games are loaded in as cogs, when started, commands only become available then
    \newpage
  \item \textbf{Secret Hitler}
  \item The President is also a server role, while chanccelor is only a backend variable
  \item last president and chancellor is a 2-Tupel of players
  \item for who can be become next chancellor check this tupel, or if below 5 players are alive,
    check only last chancellor
  \item Secret Hiler roles are stored, as a array, in an order, that the top x roles are always
    the roles needed, to play with x players
  \item When the game starts with x players, it takes the x highest roles, shuffles them, and
    attributes them to a player, sending them their role in private as a picture
  \item The role attribute is saved in a dict, which can be looked up when necessary, with player ID
    as key
  \item When investigating, Tag a player in server chat, and get the party membership as picture in
    private
  \item The Presidents tags a player @player to announce them as chancellor
  \item The vote is being cast in private chat, the result is being published, when all votes came
    in, into the Secret Hitler Chat, as a Table
  \item Passed policies are stored as a int array (2-Tupel) with one number being passed fascist,
    and one being liberal
  \item the game board, and special actions, are being triggered by comparing with this array
  \item {\color{red}Carefull with when policies can be rejected}
  \item The game board will be returned into the Secret Hitler Chat as a picture
\end{itemize}

\end{document}
